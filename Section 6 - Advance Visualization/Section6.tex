\documentclass[]{article}
\usepackage{lmodern}
\usepackage{amssymb,amsmath}
\usepackage{ifxetex,ifluatex}
\usepackage{fixltx2e} % provides \textsubscript
\ifnum 0\ifxetex 1\fi\ifluatex 1\fi=0 % if pdftex
  \usepackage[T1]{fontenc}
  \usepackage[utf8]{inputenc}
\else % if luatex or xelatex
  \ifxetex
    \usepackage{mathspec}
  \else
    \usepackage{fontspec}
  \fi
  \defaultfontfeatures{Ligatures=TeX,Scale=MatchLowercase}
\fi
% use upquote if available, for straight quotes in verbatim environments
\IfFileExists{upquote.sty}{\usepackage{upquote}}{}
% use microtype if available
\IfFileExists{microtype.sty}{%
\usepackage{microtype}
\UseMicrotypeSet[protrusion]{basicmath} % disable protrusion for tt fonts
}{}
\usepackage[margin=1in]{geometry}
\usepackage{hyperref}
\hypersetup{unicode=true,
            pdftitle={Section 6},
            pdfauthor={Cassie E. Skipper},
            pdfborder={0 0 0},
            breaklinks=true}
\urlstyle{same}  % don't use monospace font for urls
\usepackage{graphicx,grffile}
\makeatletter
\def\maxwidth{\ifdim\Gin@nat@width>\linewidth\linewidth\else\Gin@nat@width\fi}
\def\maxheight{\ifdim\Gin@nat@height>\textheight\textheight\else\Gin@nat@height\fi}
\makeatother
% Scale images if necessary, so that they will not overflow the page
% margins by default, and it is still possible to overwrite the defaults
% using explicit options in \includegraphics[width, height, ...]{}
\setkeys{Gin}{width=\maxwidth,height=\maxheight,keepaspectratio}
\IfFileExists{parskip.sty}{%
\usepackage{parskip}
}{% else
\setlength{\parindent}{0pt}
\setlength{\parskip}{6pt plus 2pt minus 1pt}
}
\setlength{\emergencystretch}{3em}  % prevent overfull lines
\providecommand{\tightlist}{%
  \setlength{\itemsep}{0pt}\setlength{\parskip}{0pt}}
\setcounter{secnumdepth}{0}
% Redefines (sub)paragraphs to behave more like sections
\ifx\paragraph\undefined\else
\let\oldparagraph\paragraph
\renewcommand{\paragraph}[1]{\oldparagraph{#1}\mbox{}}
\fi
\ifx\subparagraph\undefined\else
\let\oldsubparagraph\subparagraph
\renewcommand{\subparagraph}[1]{\oldsubparagraph{#1}\mbox{}}
\fi

%%% Use protect on footnotes to avoid problems with footnotes in titles
\let\rmarkdownfootnote\footnote%
\def\footnote{\protect\rmarkdownfootnote}

%%% Change title format to be more compact
\usepackage{titling}

% Create subtitle command for use in maketitle
\newcommand{\subtitle}[1]{
  \posttitle{
    \begin{center}\large#1\end{center}
    }
}

\setlength{\droptitle}{-2em}

  \title{Section 6}
    \pretitle{\vspace{\droptitle}\centering\huge}
  \posttitle{\par}
    \author{Cassie E. Skipper}
    \preauthor{\centering\large\emph}
  \postauthor{\par}
      \predate{\centering\large\emph}
  \postdate{\par}
    \date{September 18, 2020}


\begin{document}
\maketitle

\subsection{Grammar of Graphics - ggplot2 - S6,
L60}\label{grammar-of-graphics---ggplot2---s6-l60}

Grammar of Graphics: Components of/Layers of Information in a Plot 1.
Data 2. Aesthetics 3. Geometrics 4. Statistics 5. Facets 6. Coordinates
7. Theme

---------------- What is a Factor? - S6, L61

getwd() setwd(``\textasciitilde{}/R Programming A-Z/Section 6 - Advance
Visualization/SuperDataScience files'') getwd()

movies \textless{}- read.csv(``P2-Movie-Ratings.csv'') head(movies)
colnames(movies) \textless{}- c(``Film'', ``Genre'', ``CriticRating'',
``AudienceRating'', ``BudgetMillions'', ``Year'') head(movies)
tail(movies) str(movies) check the variable (column) types (factor vs
integer)

What is a Factor? Factors are categorical variables; factors are the way
R works with categorical variables levels and numbers; R looks through
your column finds all of the different words that exist and assigns each
word a number

summary(movies) we want Year to be a categorical variable aka factor

How to convert a non-factor numeric variable --\textgreater{} factor
factor(movies\(Year) convert into factor movies\)Year \textless{}-
factor(movies\$Year) assign column in the dataset the new factored
vector summary(movies) Year is now a factor str(movies) confirm that
Year is now a factor

---------------- Aesthetics - S6, L62

Aesthetics - how your data maps to what you want to see

library(ggplot2)

last time we used qplot() - q stands for quick plot ggplot is the
better, more in-depth/customizable option ggplot(data=movies,
aes(x=CriticRating, y=AudienceRating)) if you run just this, you will
get an empty plot

add geometry ggplot(data=movies, aes(x=CriticRating, y=AudienceRating))
+ geom\_point()

add color ggplot(data=movies, aes(x=CriticRating, y=AudienceRating,
colour=Genre)) + geom\_point()

add size ggplot(data=movies, aes(x=CriticRating, y=AudienceRating,
colour=Genre, size=Genre)) + geom\_point() Warning message: Using size
for a discrete variable is not advised.

add size - better way ggplot(data=movies, aes(x=CriticRating,
y=AudienceRating, colour=Genre, size=BudgetMillions)) + geom\_point()
\textgreater{}\textgreater{}\textgreater{} This is 1 (but we will
further improve it later)

---------------- Plotting with Layers - S6, L63

p \textless{}- ggplot(data=movies, aes(x=CriticRating, y=AudienceRating,
colour=Genre, size=BudgetMillions))

point p + geom\_point()

lines p + geom\_line()

multiple layers p + geom\_point() + geom\_line() p + geom\_line() +
geom\_point() points on top of lines

---------------- Overriding Aesthetics - S6, L64

q \textless{}- ggplot(data=movies, aes(x=CriticRating, y=AudienceRating,
colour=Genre, size=BudgetMillions))

add geom layer q + geom\_point()

overriding aesthetics - example 1 q +
geom\_point(aes(size=CriticRating))

overriding aesthetics - example 2 q +
geom\_point(aes(colour=BudgetMillions))

q remains the same q + geom\_point()

override x and y - example 3 q + geom\_point(aes(x=BudgetMillions))
doesn't change the x-axis label q + geom\_point(aes(x=BudgetMillions)) +
xlab(``Budget Millions \($\)'')

reduce line size - example 4 q + geom\_line(size=1) + geom\_point()\\
Mapping vs.~Setting: aes() is mapping; assigning size=1 is just setting
so you don't need aes()

---------------- Mapping vs.~Setting - S6, L65

q \textless{}- ggplot(data=movies, aes(x=CriticRating, y=AudienceRating,
colour=Genre, size=BudgetMillions))

override x and y - example 3 q + geom\_point(aes(x=BudgetMillions)) +
xlab(``Budget Millions \($\)'')

r \textless{}- ggplot(data=movies, aes(x=CriticRating,
y=AudienceRating)) r + geom\_point()

Add colour 1. Mapping (what we've done so far) r +
geom\_point(aes(colour=Genre)) mapping color to Genre variable 2.
Setting r + geom\_point(colour=``DarkGreen'') setting color at DarkGreen
ERROR: r + geom\_point(aes(colour=``DarkGreen''))\\
you're actually mapping DarkGreen as a new variable - not as a color!!

\begin{enumerate}
\def\labelenumi{\arabic{enumi}.}
\tightlist
\item
  Mapping r + geom\_point(aes(size=BudgetMillions))
\item
  Setting r + geom\_point(size=10) ERROR: r + geom\_point(aes(size=10))
  r sees size=10 as a variable
\end{enumerate}

---------------- Histograms and Density Charts - S6, L66

Histograms s \textless{}- ggplot(data=movies, aes(x=BudgetMillions)) s +
geom\_histogram(binwidth=10) count (y-axis) is generated by r; it's a
statistic

add colour s + geom\_histogram(binwidth=10, fill=``Green'') setting s +
geom\_histogram(binwidth=10, aes(fill=Genre)) mapping

add a border s + geom\_histogram(binwidth=10, aes(fill=Genre),
colour=``Black'') color of fill is mapped to Genre color of the borders
is set to Black \textgreater{}\textgreater{}\textgreater{} Chart 3 (we
will improve it later)

Density Charts sometimes you may need them s +
geom\_density(aes(fill=Genre)) s + geom\_density(aes(fill=Genre),
position=``stack'') another option: use + geom\_area(position =
``stack'') if you have x and y vars: s + geom\_density(aes(fill=Genre))
+ geom\_area(position=``stack'')

---------------- Starting Layer Tips - S6, L67

t \textless{}- ggplot(data=movies, aes(x=AudienceRating)) t +
geom\_histogram(binwidth=10, fill=``White'', colour=``Blue'')\\
fill and border colors are set (not mapped)

another way to achieve this\^{} plot (faster to adapt) t \textless{}-
ggplot(data=movies) t + geom\_histogram(binwidth=10,
aes(x=AudienceRating), fill=``White'', colour=``Blue'')
\textgreater{}\textgreater{}\textgreater{} Chart 4

t + geom\_histogram(binwidth=10, aes(x=CriticRating), fill=``White'',
colour=``Blue'') \textgreater{}\textgreater{}\textgreater{} Chart 5

t \textless{}- ggplot() skeleton plot

---------------- Statistical Transformations - S6, L68

library(ggplot2)

?geom\_smooth()

u \textless{}- ggplot(data=movies, aes(x=CriticRating, y=AudienceRating,
colour=Genre)) u + geom\_point() + geom\_smooth() u + geom\_point() +
geom\_smooth(fill=NA)

Boxplots u \textless{}- ggplot(data=movies, aes(x=Genre,
y=AudienceRating, colour=Genre)) u + geom\_boxplot() u +
geom\_boxplot(size=1.2) see bonus tutorial for boxplots u +
geom\_boxplot(size=1.2) + geom\_point() tip / hack: u +
geom\_boxplot(size=1.2) + geom\_jitter() geom\_jitter() puts the dots on
the boxplots to help visualize the data another way: u + geom\_jitter()
+ geom\_boxplot(size=1.2) dots behind the boxplots another example: u +
geom\_jitter() + geom\_boxplot(size=1.2, alpha=0.5) alpha controls
transparency (0-1) boxplots are 50\% transparent
\textgreater{}\textgreater{}\textgreater{} Chart 6

Challenge t \textless{}- ggplot(data=movies, aes(x=Genre,
y=CriticRating, colour=Genre)) t + geom\_jitter() +
geom\_boxplot(size=1.2, alpha=0.5)

---------------- Using Facets - S6, L69

v \textless{}- ggplot(data=movies, aes(x=BudgetMillions)) v +
geom\_histogram(binwdith=10) v + geom\_histogram(binwdith=10,
aes(fill=Genre), colour=``Black'')

Facets: allow you to create lots of charts histogram for each genre

uniform scales: v + geom\_histogram(binwdith=10, aes(fill=Genre),
colour=``Black'') + facet\_grid(Genre\textasciitilde{}.) rows=Genre
facet\_grid(row\textasciitilde{}column) notice that the Comedy genre
takes up the entire scale all scales are uniform, unless specified

different scales: v + geom\_histogram(binwdith=10, aes(fill=Genre),
colour=``Black'') + facet\_grid(Genre\textasciitilde{}.,
scales=``free'') scales=free removes scale uniformity

Scatterplots: w \textless{}- ggplot(data=movies, aes(x=CriticRating,
y=AudienceRating, colour=Genre)) w + geom\_point(size=3)

scatterplot facets: w + geom\_point(size=3) +
facet\_grid(Genre\textasciitilde{}.)

w + geom\_point(size=3) + facet\_grid(.\textasciitilde{}Year)

w + geom\_point(size=3) + facet\_grid(Genre\textasciitilde{}Year)

w + geom\_point(size=3) + geom\_smooth() +
facet\_grid(Genre\textasciitilde{}Year)

w + geom\_point(aes(size=BudgetMillions)) + geom\_smooth() +
facet\_grid(Genre\textasciitilde{}Year)
\textgreater{}\textgreater{}\textgreater{} Chart 1 (but stil will
improve) the geom\_smooth is stretching the axes; we need to zoom in
(see L70)

---------------- Coordinates - S6, L70

limits zooming in \& out

w + geom\_point(aes(size=BudgetMillions)) + geom\_smooth() +
facet\_grid(Genre\textasciitilde{}Year)

m \textless{}- ggplot(data=movies, aes(x=CriticRating, y=AudienceRating,
size=BudgetMillions, colour=Genre)) m + geom\_point()

zoom in on top right quadrant m + geom\_point() + xlim(50,100) +
ylim(50,100)

\^{}this won't always work well n \textless{}- ggplot(data=movies,
aes(x=BudgetMillions)) n + geom\_histogram(binwidth=10, aes(fill=Genre),
colour=``Black'') cut if off up to 50 n + geom\_histogram(binwidth=10,
aes(fill=Genre), colour=``Black'') + ylim(0,50) this cuts off any data
that goes to 50 or beyond

instead of this\^{}, Zoom in using coord\_cartesian(): n +
geom\_histogram(binwidth=10, aes(fill=Genre), colour=``Black'') +
coord\_cartesian(ylim=c(0,50))

back to improving Chart 1; zoom in w +
geom\_point(aes(size=BudgetMillions)) + geom\_smooth() +
facet\_grid(Genre\textasciitilde{}Year) +
coord\_cartesian(ylim=c(0,100))
\textgreater{}\textgreater{}\textgreater{} Chart \$1

---------------- Perfecting by Adding Themes - S6, L71

Themes: includes all non-data ink ex: position of labels and legends,
color of background, text

improving s (Chart 3) from L66 Histograms \& Density Charts o
\textless{}- ggplot(data=movies, aes(x=BudgetMillions)) h \textless{}- o
+ geom\_histogram(binwidth=10, aes(fill=Genre), colour=``Black'') h

axes labels h + xlab(``Money Axis'') + ylab(``Number of Movies'')

label formatting h + xlab(``Money Axis'') + ylab(``Number of Movies'') +
theme(axis.title.x = element\_text(colour=``DarkGreen'', size=15),
axis.title.y = element\_text(colour=``Red'', size=15))

tick mark formatting h + xlab(``Money Axis'') + ylab(``Number of
Movies'') + theme(axis.title.x = element\_text(colour=``DarkGreen'',
size=15), axis.title.y = element\_text(colour=``Red'', size=15),
axis.text.x = element\_text(size=10), axis.text.y =
element\_text(size=10))

?theme()

legend formatting h + xlab(``Money Axis'') + ylab(``Number of Movies'')
+ theme(axis.title.x = element\_text(colour=``DarkGreen'', size=15),
axis.title.y = element\_text(colour=``Red'', size=15), axis.text.x =
element\_text(size=10), axis.text.y = element\_text(size=10),

\begin{verbatim}
    legend.title = element_text(size=15),
    legend.text = element_text(size=10),
    legend.position = c(1,1),   can pass 0 (start of axis) or 1 (end of axis)
    legend.justification = c(1.05,1.05))  anchors the legend      
\end{verbatim}

import fonts install.packages(``extrafont'') library(extrafont)
font\_import(``C:/Windows/Fonts'') loadfonts(device = ``win'')

plot title h + xlab(``Money Axis'') + ylab(``Number of Movies'') +
ggtitle(``Movie Budget Distribution'') + theme(axis.title.x =
element\_text(colour=``DarkGreen'', size=15), axis.title.y =
element\_text(colour=``Red'', size=15), axis.text.x =
element\_text(size=10), axis.text.y = element\_text(size=10),

\begin{verbatim}
    legend.title = element_text(size=15),
    legend.text = element_text(size=10),
    legend.position = c(1,1),   can pass 0 (start of axis) or 1 (end of axis)
    legend.justification = c(1.05,1.05),  anchors the legend
    
    plot.title = element_text(colour="DarkBlue", 
                              size=20, 
                              family="Courier New",
                              hjust=0.5))
\end{verbatim}


\end{document}
